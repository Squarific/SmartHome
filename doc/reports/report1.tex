\documentclass[11pt]{article}

\usepackage[dutch]{babel}
\usepackage{fullpage}
\usepackage{xcolor,listings}
\usepackage{graphicx}
\usepackage{float}

\lstdefinestyle{SQL}{
   language=SQL,
   showspaces=false,
   basicstyle=\ttfamily,
   numbers=none,
   numberstyle=\tiny,
   commentstyle=\color{gray},
   breaklines=true
}

%Gummi|065|=)
\title{\textbf{Rapport 1}}
\author{Filip Smets\\
		Nisse Strauven\\
		Jens Bijtebier\\
		Thanh Danh Le\\
		David Danssaert}
\date{\today}
\begin{document}

\maketitle

\tableofcontents

\section{Status}
\section{Design}
  \subsection{Keuzes}
  \subsubsection{Dependencies}
  \paragraph{Clientside}
  \begin{itemize}
  \item node.js
  \item npm
  \item ember-cli (via npm)
  \item bower (via npm)
  \item ember-material-design? (via ember, not yet included)
  \end{itemize}

  \paragraph{Serverside}
  \begin{itemize}
  \item python 3
  \item django (via pip)
  \item djangorestframework (via pip)
  \item markdown (via pip)
  \item django-filter (via pip)
  \end{itemize}
  \subsection{Database Schema}
  Zie \textbf{figuur \ref{fig:erd}}.
  \begin{figure}[H]
  \centering
    \includegraphics[width=\textwidth]{../erd/SmartHome_ERD.png}
  \caption{Het `Entity Relation Diagram'.}
  \label{fig:erd}
  \end{figure}
  
  \subsection{UML Schema}
  Zie \textbf{figuur \ref{fig:uml}}.
  \begin{figure}[H]
  \centering
    \includegraphics[width=\textwidth]{img/SmartHome_deployment_diagram.png}
  \caption{Deployment diagram voor server (links) en client (rechts).}
  \label{fig:uml}
  \end{figure}
\section{Product}
  \subsection{Basisvereisten}
  \subsection{Extra functionaliteit}
\section{Planning}
\section{Appendix}
  \subsection{SQL Queries}
  \subsubsection{Households}
\begin{lstlisting}[style=SQL]
/* list all households */
SELECT homes.id, homes.owner_id, homes.name, homes.country, homes.city, homes.zipcode, homes.street, homes.house_number, homes.date_added
FROM homes;

/* retrieve household by home ID */
SELECT homes.id, homes.owner_id, homes.name, homes.country, homes.city, homes.zipcode, homes.street, homes.house_number, homes.date_added
FROM homes
WHERE homes.id = $ID;

/* retrieve household by user ID */
SELECT homes.id, homes.owner_id, homes.name, homes.country, homes.city, homes.zipcode, homes.street, homes.house_number, homes.date_added
FROM homes
INNER JOIN users_homes
ON (homes.id = users_homes.home_id)
WHERE users_homes.user_id = $ID;
\end{lstlisting}

  \subsubsection{Sensors}
\begin{lstlisting}[style=SQL]
/* list all sensors */
SELECT sensors.id, sensors.home_id, sensors.name, sensors.description, sensors.power_unit, sensors.date_created
FROM sensors;

/* retrieve sensor by id */
SELECT sensors.id, sensors.home_id, sensors.name, sensors.description, sensors.power_unit, sensors.date_created
FROM sensors
WHERE sensors.id = $ID;

/* retreive sensors by tag id */
SELECT sensors.id, sensors.home_id, sensors.name, sensors.description, sensors.power_unit, sensors.date_created
FROM sensors
INNER JOIN sensors_tags
ON (sensors.id = sensors_tags.sensor_id)
WHERE sensors_tags.tag_id = $ID;
\end{lstlisting}

  \subsubsection{Sensor Tags}
\begin{lstlisting}[style=SQL]
/* list all tags */
SELECT tags.id, tags.name, tags.description
FROM tags;

/* retrieve tag by id */
SELECT tags.id, tags.name, tags.description
FROM tags
WHERE tags.id = $ID;

/* retrieve tags for a given sensor (by id) */
SELECT tags.id, tags.name, tags.description
FROM tags
INNER JOIN sensors_tags
ON (tags.id = sensors_tags.tag_id)
WHERE sensors_tags.sensor_id = $ID;
\end{lstlisting}

  \subsubsection{Sensor Data}
\begin{lstlisting}[style=SQL]
/* list the sum total usage of all sensors for each minute of today for a given user */
SELECT recent_data.timestamp, sensors.home_id AS home_id, homes.name AS home_name, SUM(recent_data.usage) AS usage
FROM recent_data
INNER JOIN sensors
ON (recent_data.sensor_id = sensors.id)
INNER JOIN homes ON (sensors.home_id = homes.id)
WHERE (homes.owner_id = $ID AND recent_data.timestamp >= DATE_SUB(NOW(), INTERVAL 1 DAY) AND recent_data.timestamp < NOW() )
GROUP BY recent_data.timestamp, sensors.home_id, homes.name
ORDER BY home_id ASC, recent_data.timestamp ASC;

/* Aggregate all minutely data from today into one record for all sensors */
INSERT INTO daily_data (sensor_id, timestamp, usage, n_measurements)
SELECT recent_data.sensor_id, CONCAT(DATE(NOW()), ' 00:00:00') AS new_date, AVG(recent_data.usage), SUM(recent_data.n_measurements)
FROM recent_data
WHERE ( recent_data.timestamp >= new_date AND recent_data.timestamp < DATE_ADD(new_date, INTERVAL 1 DAY) )
GROUP BY recent_data.sensor_id

\end{lstlisting}  
\end{document}
