\documentclass[11pt]{article}

\usepackage[dutch]{babel}
\usepackage{fullpage}
\usepackage{xcolor,listings}
\usepackage{graphicx}
\usepackage{float}

\lstdefinestyle{SQL}{
   language=SQL,
   showspaces=false,
   basicstyle=\ttfamily,
   numbers=none,
   numberstyle=\tiny,
   commentstyle=\color{gray},
   breaklines=true
}

%Gummi|065|=)
\title{\textbf{Programming Project Databases - Rapport 2}\\
		Groep 4}
\author{Jens Bijtebier\\
		David Danssaert\\
		Thanh Danh Le\\
		Filip Smets\\
		Nisse Strauven}
\date{\today}
\begin{document}

\maketitle

%\tableofcontents

\section{Status}
  \subsection{Homepage}
	\subsubsection{Appealing to the end user}
		The Homepage is where the end user will be when he or she first visits SmartHome.
		Since first impressions are important, we made sure to make the homepage look appealing,
		and added some example graphs to showcase the features of our project.\\

		(Thanh, Jens)

  \subsection{Translation}
	\subsubsection{Dutch - English}
		The footer features a Dutch and English flag, used to switch languages.
		Everything on the website is translated in both languages, and they can be switched on the spot, at any time.

		(Filip, Jens)

  \subsection{Login and Registration}
	\subsubsection{Simple and elegant}
		We have added both the register and login button to the top-right of the screen.
		Both buttons open a pop-over dialog, which makes for a smooth transition.
		This is one of the reasons we went for React in combination with Material-UI.
		Everything happens on one page. No refreshing and unnecessary loading times needed.\\

		(Filip)

  \subsection{Households and Sensors}
	\subsubsection{Creation}
		Although the simulation software doesn't really support this, the user can create both households and sensors.
		Creating a household opens a new dialog where the user can fill in the name, country, city, zip code, street, and housenumber.
		Likewise, creating a sensor opens another dialog where the user can provide a name, household, description, power unit, and tags.\\

		(Thanh, Jens)

	\subsubsection{Viewing}
		The "View Households" menu changes the layout of the page so that it now displays a card for every household.
		The graph type can be chosen by the user (Line, Bar, Radar).
		It is also possible to change how the usage is shown (price in euros, or the original format).\\

		(Filip)

	\subsubsection{Filtering}
		Within the Household cards, the user can select which sensor should be shown (a combination of sensors is also possible).
		It is also possible to specify a time period and graph type, allowing for maximum flexibility.\\

		(Filip)

	\subsubsection{Tags and Editing}
		Household cards also contain a button that can be used to edit a sensor's tags.
		This button opens a new dialog where tags can be added and removed to the user's liking.\\

		(Jens)

  \subsection{Social Aspect}
	\subsubsection{Friends}
		Using the friend search feature, it is possible to find other people by name and send them a friend request.
		Users need to be friends before they can see each other's posts and shared graphs.\\

		(Thanh)

	\subsubsection{Notifications}
		Users can find any pending friend requests in their notifications menu, where they can choose to approve said request.
		Once the request is approved, they will be able to see posts and compare their energy consumption using the Ranking feature.\\

		(Jens)

	\subsubsection{Wall}
		The Wall is where all the user's friends' posts and shared graphs will be displayed,
		much like the activity feed on the Facebook homepage.
		Shared graphs can still be viewed using a graph type chosen by the end user.\\

		There is also a button that opens a dialog for posting a new message.\\

		(Jens)

	\subsubsection{Sharing Graphs}
		Every card in the "View Household" menu has a "Share Graph" button that opens up a new dialog where the graph can be shared,
		along with a textfield for any additional info the user might want provide.

		(Jens)
		
  \subsection{Back end}
	\subsubsection{IK DENK DAT DAVID DIT HET BEST EVEN INVULT?}
	\subsubsection{Meerdere subsubsections om het proper te houden pls}

  \subsection{Extra Functionalities}
	\subsubsection{Ranking}
		The first extra feature we have decided to add is the possibility to see a ranking among friends.
		This ranking would show which one of your friends on SmartHome has the lowest energy consumption.
		We feel that this would provide an extra incentive to try and decrease your usage.\\

		It is possible to rank either by total usage, or average usage per household.\\

		(Nisse)

	\subsubsection{Clustering}
		WILT FILIP OF DAVID DIT EVEN INVULLEN? KKTHXBYE

\section{Design}
  \subsection{Client}
	\subsubsection{React}
		We have chosen to use React for our client's UI. React describes itself as "the V in MVC",
		which is exactly what we intend to use it for.\\
		
		React's component based design allows us to construct new components (such as the graphs, household cards, notifications, and so on),
		and then use these components to create so called PageComponents. These allow us to switch pages without the need to refresh the browser,
		ensuring maximum smoothness.

	\subsubsection{Material-UI}
		Google's widely used design standard, Material, is used everywhere. Android, along with many smartphone apps and websites use Material.
		This means the user will recognize the interface, making the UI feel familiar.
		Furthermore, the interface will react exactly the way the end user expects it to, making Material the best possible option.

		The Material-UI Library for react incorporates these design standards in simple components such as appbars, sidebars, and more.
		This made it possible for us to create a responsive and recognizable design that will do exactly what the user wants it to.
		
	\subsubsection{ChartJs}
		Graphs are the core of SmartHome, so their responsiveness and readability were top priority for us.
		We have opted to use ChartJs, a React library that provides us with easiliy customizable chart components,
		which we have incorporated in the household cards. 
		
  \subsection{Server}
	\subsubsection{Django}
		DAVID VUL HIER WAT MOOIE WOORDEN IN PLS

  \subsection{REST API}
	\subsubsection{Retrieving data from the server}
		The communication between client and server happens using a well defined REST api. This splits the logic well, and ensures seperation of concerns.
		Client side, we use plain XMLHttpRequests wrapped in a small homebrew library. Server side, Django provides an interface to make it easy to handle the requests.

  \subsection{Database Schema}
  %Zie \textbf{figuur \ref{fig:erd}}.
  \begin{figure}[H]
  \centering
    \includegraphics[width=\textwidth]{../erd/SmartHome_ERD.png}
  \caption{Het `Entity Relation Diagram'.}
  \label{fig:erd}
  \end{figure}
  
  \subsection{UML Schema}
  %Zie \textbf{figuur \ref{fig:uml}}.
  \begin{figure}[H]
  \centering
    \includegraphics[width=\textwidth]{img/SmartHome_deployment_diagram.png}
  \caption{Deployment diagram voor server (links) en client (rechts).}
  \label{fig:uml}
  \end{figure}

\section{Product}
	This section contains a short synopsis of all the required and extra features that we have implemented.
	For more information about a certain feature, please refer to the Status section.

  \subsection{Basisvereisten}
	All requirements have been met. We have implemented a REST interface, to retrieve the sensor data from the server.
	This data can then be shown in the "View Household" menu, and the user can select one or more sensors and a time period to see exactly what he or she wants.
	The user can register, log in, create sensors and households. Sensors have editable tags, a name, a power unit, a description, and a household.
	Households also have all required attributes.\\

	Translations are part of the interface, so that languages can be switched at any time.\\

	On the social side, it is possible to search for other users, and send them a friendrequest.
	Once they have accepted the request in their notifications menu, it will be possible to see each other's posts.
	Posts can either be posted using the wall, resulting in a plain text post, or graphs can be shared.
	When the end users shares a graph, it is possible to add a message to the graph in order to provide some extra information.\\

	An admin interface is accessible for users that are marked as staff.\\

  \subsection{Extra functionaliteit}
	CLUSTERING NOG INVULLEN

	We have also added a ranking feature that allows the user to compare energy usage amongst their friends.
	Users that have a lower average consumption, will rank higher. This promotes saving energy, which is essentially what SmartHome is all about.

\section{Appendix}
  \subsection{SQL Queries}
  \subsubsection{Households}
\begin{lstlisting}[style=SQL]
/* list all households */
SELECT homes.id, homes.owner_id, homes.name, homes.country, homes.city, homes.zipcode, homes.street, homes.house_number, homes.date_added
FROM homes;

/* retrieve household by home ID */
SELECT homes.id, homes.owner_id, homes.name, homes.country, homes.city, homes.zipcode, homes.street, homes.house_number, homes.date_added
FROM homes
WHERE homes.id = $ID;

/* retrieve household by user ID */
SELECT homes.id, homes.owner_id, homes.name, homes.country, homes.city, homes.zipcode, homes.street, homes.house_number, homes.date_added
FROM homes
INNER JOIN users_homes
ON (homes.id = users_homes.home_id)
WHERE users_homes.user_id = $ID;
\end{lstlisting}

  \subsubsection{Sensors}
\begin{lstlisting}[style=SQL]
/* list all sensors */
SELECT sensors.id, sensors.home_id, sensors.name, sensors.description, sensors.power_unit, sensors.date_created
FROM sensors;

/* retrieve sensor by id */
SELECT sensors.id, sensors.home_id, sensors.name, sensors.description, sensors.power_unit, sensors.date_created
FROM sensors
WHERE sensors.id = $ID;

/* retreive sensors by tag id */
SELECT sensors.id, sensors.home_id, sensors.name, sensors.description, sensors.power_unit, sensors.date_created
FROM sensors
INNER JOIN sensors_tags
ON (sensors.id = sensors_tags.sensor_id)
WHERE sensors_tags.tag_id = $ID;
\end{lstlisting}

  \subsubsection{Sensor Tags}
\begin{lstlisting}[style=SQL]
/* list all tags */
SELECT tags.id, tags.name, tags.description
FROM tags;

/* retrieve tag by id */
SELECT tags.id, tags.name, tags.description
FROM tags
WHERE tags.id = $ID;

/* retrieve tags for a given sensor (by id) */
SELECT tags.id, tags.name, tags.description
FROM tags
INNER JOIN sensors_tags
ON (tags.id = sensors_tags.tag_id)
WHERE sensors_tags.sensor_id = $ID;
\end{lstlisting}

  \subsubsection{Sensor Data}
\begin{lstlisting}[style=SQL]
/* list the sum total usage of all sensors for each minute of today for a given user */
SELECT recent_data.timestamp, sensors.home_id AS home_id, homes.name AS home_name, SUM(recent_data.usage) AS usage
FROM recent_data
INNER JOIN sensors
ON (recent_data.sensor_id = sensors.id)
INNER JOIN homes ON (sensors.home_id = homes.id)
WHERE (homes.owner_id = $ID AND recent_data.timestamp >= DATE_SUB(NOW(), INTERVAL 1 DAY) AND recent_data.timestamp < NOW() )
GROUP BY recent_data.timestamp, sensors.home_id, homes.name
ORDER BY home_id ASC, recent_data.timestamp ASC;

/* Aggregate all minutely data from today into one record for all sensors */
INSERT INTO daily_data (sensor_id, timestamp, usage, n_measurements)
SELECT recent_data.sensor_id, CONCAT(DATE(NOW()), ' 00:00:00') AS new_date, AVG(recent_data.usage), SUM(recent_data.n_measurements)
FROM recent_data
WHERE ( recent_data.timestamp >= new_date AND recent_data.timestamp < DATE_ADD(new_date, INTERVAL 1 DAY) )
GROUP BY recent_data.sensor_id

\end{lstlisting}

\subsection{SQL Database Structure}
\begin{lstlisting}[style=SQL]
--
-- Tabelstructuur voor tabel `account_emailaddress`
--

CREATE TABLE IF NOT EXISTS `account_emailaddress` (
  `id` int(11) NOT NULL,
  `email` varchar(254) NOT NULL,
  `verified` tinyint(1) NOT NULL,
  `primary` tinyint(1) NOT NULL,
  `user_id` int(11) NOT NULL
) ENGINE=InnoDB DEFAULT CHARSET=latin1;

-- --------------------------------------------------------

--
-- Tabelstructuur voor tabel `account_emailconfirmation`
--

CREATE TABLE IF NOT EXISTS `account_emailconfirmation` (
  `id` int(11) NOT NULL,
  `created` datetime(6) NOT NULL,
  `sent` datetime(6) DEFAULT NULL,
  `key` varchar(64) NOT NULL,
  `email_address_id` int(11) NOT NULL
) ENGINE=InnoDB DEFAULT CHARSET=latin1;

-- --------------------------------------------------------

--
-- Tabelstructuur voor tabel `authtoken_token`
--

CREATE TABLE IF NOT EXISTS `authtoken_token` (
  `key` varchar(40) NOT NULL,
  `created` datetime(6) NOT NULL,
  `user_id` int(11) NOT NULL
) ENGINE=InnoDB DEFAULT CHARSET=latin1;

-- --------------------------------------------------------

--
-- Tabelstructuur voor tabel `auth_group`
--

CREATE TABLE IF NOT EXISTS `auth_group` (
  `id` int(11) NOT NULL,
  `name` varchar(80) NOT NULL
) ENGINE=InnoDB DEFAULT CHARSET=latin1;

-- --------------------------------------------------------

--
-- Tabelstructuur voor tabel `auth_group_permissions`
--

CREATE TABLE IF NOT EXISTS `auth_group_permissions` (
  `id` int(11) NOT NULL,
  `group_id` int(11) NOT NULL,
  `permission_id` int(11) NOT NULL
) ENGINE=InnoDB DEFAULT CHARSET=latin1;

-- --------------------------------------------------------

--
-- Tabelstructuur voor tabel `auth_permission`
--

CREATE TABLE IF NOT EXISTS `auth_permission` (
  `id` int(11) NOT NULL,
  `name` varchar(255) NOT NULL,
  `content_type_id` int(11) NOT NULL,
  `codename` varchar(100) NOT NULL
) ENGINE=InnoDB AUTO_INCREMENT=76 DEFAULT CHARSET=latin1;

--
-- Gegevens worden geëxporteerd voor tabel `auth_permission`
--

INSERT INTO `auth_permission` (`id`, `name`, `content_type_id`, `codename`) VALUES
(1, 'Can add log entry', 1, 'add_logentry'),
(2, 'Can change log entry', 1, 'change_logentry'),
(3, 'Can delete log entry', 1, 'delete_logentry'),
(4, 'Can add permission', 2, 'add_permission'),
(5, 'Can change permission', 2, 'change_permission'),
(6, 'Can delete permission', 2, 'delete_permission'),
(7, 'Can add group', 3, 'add_group'),
(8, 'Can change group', 3, 'change_group'),
(9, 'Can delete group', 3, 'delete_group'),
(10, 'Can add user', 4, 'add_user'),
(11, 'Can change user', 4, 'change_user'),
(12, 'Can delete user', 4, 'delete_user'),
(13, 'Can add content type', 5, 'add_contenttype'),
(14, 'Can change content type', 5, 'change_contenttype'),
(15, 'Can delete content type', 5, 'delete_contenttype'),
(16, 'Can add session', 6, 'add_session'),
(17, 'Can change session', 6, 'change_session'),
(18, 'Can delete session', 6, 'delete_session'),
(19, 'Can add site', 7, 'add_site'),
(20, 'Can change site', 7, 'change_site'),
(21, 'Can delete site', 7, 'delete_site'),
(22, 'Can add token', 8, 'add_token'),
(23, 'Can change token', 8, 'change_token'),
(24, 'Can delete token', 8, 'delete_token'),
(25, 'Can add email address', 9, 'add_emailaddress'),
(26, 'Can change email address', 9, 'change_emailaddress'),
(27, 'Can delete email address', 9, 'delete_emailaddress'),
(28, 'Can add email confirmation', 10, 'add_emailconfirmation'),
(29, 'Can change email confirmation', 10, 'change_emailconfirmation'),
(30, 'Can delete email confirmation', 10, 'delete_emailconfirmation'),
(31, 'Can add social application', 11, 'add_socialapp'),
(32, 'Can change social application', 11, 'change_socialapp'),
(33, 'Can delete social application', 11, 'delete_socialapp'),
(34, 'Can add social account', 12, 'add_socialaccount'),
(35, 'Can change social account', 12, 'change_socialaccount'),
(36, 'Can delete social account', 12, 'delete_socialaccount'),
(37, 'Can add social application token', 13, 'add_socialtoken'),
(38, 'Can change social application token', 13, 'change_socialtoken'),
(39, 'Can delete social application token', 13, 'delete_socialtoken'),
(40, 'Can add cors model', 14, 'add_corsmodel'),
(41, 'Can change cors model', 14, 'change_corsmodel'),
(42, 'Can delete cors model', 14, 'delete_corsmodel'),
(43, 'Can add home', 15, 'add_home'),
(44, 'Can change home', 15, 'change_home'),
(45, 'Can delete home', 15, 'delete_home'),
(46, 'Can add sensor', 16, 'add_sensor'),
(47, 'Can change sensor', 16, 'change_sensor'),
(48, 'Can delete sensor', 16, 'delete_sensor'),
(49, 'Can add sensors tags', 17, 'add_sensorstags'),
(50, 'Can change sensors tags', 17, 'change_sensorstags'),
(51, 'Can delete sensors tags', 17, 'delete_sensorstags'),
(52, 'Can add tag', 18, 'add_tag'),
(53, 'Can change tag', 18, 'change_tag'),
(54, 'Can delete tag', 18, 'delete_tag'),
(55, 'Can add users homes', 19, 'add_usershomes'),
(56, 'Can change users homes', 19, 'change_usershomes'),
(57, 'Can delete users homes', 19, 'delete_usershomes'),
(58, 'Can add yearly data', 20, 'add_yearlydata'),
(59, 'Can change yearly data', 20, 'change_yearlydata'),
(60, 'Can delete yearly data', 20, 'delete_yearlydata'),
(61, 'Can add monthly data', 21, 'add_monthlydata'),
(62, 'Can change monthly data', 21, 'change_monthlydata'),
(63, 'Can delete monthly data', 21, 'delete_monthlydata'),
(64, 'Can add daily data', 22, 'add_dailydata'),
(65, 'Can change daily data', 22, 'change_dailydata'),
(66, 'Can delete daily data', 22, 'delete_dailydata'),
(67, 'Can add recent data', 23, 'add_recentdata'),
(68, 'Can change recent data', 23, 'change_recentdata'),
(69, 'Can delete recent data', 23, 'delete_recentdata'),
(70, 'Can add friend request', 24, 'add_friendrequest'),
(71, 'Can change friend request', 24, 'change_friendrequest'),
(72, 'Can delete friend request', 24, 'delete_friendrequest'),
(73, 'Can add post', 25, 'add_post'),
(74, 'Can change post', 25, 'change_post'),
(75, 'Can delete post', 25, 'delete_post');

-- --------------------------------------------------------

--
-- Tabelstructuur voor tabel `auth_user`
--

CREATE TABLE IF NOT EXISTS `auth_user` (
  `id` int(11) NOT NULL,
  `password` varchar(128) NOT NULL,
  `last_login` datetime(6) DEFAULT NULL,
  `is_superuser` tinyint(1) NOT NULL,
  `username` varchar(30) NOT NULL,
  `first_name` varchar(30) NOT NULL,
  `last_name` varchar(30) NOT NULL,
  `email` varchar(254) NOT NULL,
  `is_staff` tinyint(1) NOT NULL,
  `is_active` tinyint(1) NOT NULL,
  `date_joined` datetime(6) NOT NULL
) ENGINE=InnoDB DEFAULT CHARSET=latin1;

-- --------------------------------------------------------

--
-- Tabelstructuur voor tabel `auth_user_groups`
--

CREATE TABLE IF NOT EXISTS `auth_user_groups` (
  `id` int(11) NOT NULL,
  `user_id` int(11) NOT NULL,
  `group_id` int(11) NOT NULL
) ENGINE=InnoDB DEFAULT CHARSET=latin1;

-- --------------------------------------------------------

--
-- Tabelstructuur voor tabel `auth_user_user_permissions`
--

CREATE TABLE IF NOT EXISTS `auth_user_user_permissions` (
  `id` int(11) NOT NULL,
  `user_id` int(11) NOT NULL,
  `permission_id` int(11) NOT NULL
) ENGINE=InnoDB DEFAULT CHARSET=latin1;

-- --------------------------------------------------------

--
-- Tabelstructuur voor tabel `daily_data`
--

CREATE TABLE IF NOT EXISTS `daily_data` (
  `id` int(11) NOT NULL,
  `timestamp` datetime(6) NOT NULL,
  `usage` int(11) NOT NULL,
  `n_measurements` int(11) NOT NULL,
  `sensor_id` int(11) NOT NULL
) ENGINE=InnoDB DEFAULT CHARSET=latin1;

-- --------------------------------------------------------

--
-- Tabelstructuur voor tabel `friend_requests`
--

CREATE TABLE IF NOT EXISTS `friend_requests` (
  `id` int(11) NOT NULL,
  `status` int(11) NOT NULL,
  `date_sent` datetime(6) NOT NULL,
  `receiver_id` int(11) NOT NULL,
  `sender_id` int(11) NOT NULL
) ENGINE=InnoDB DEFAULT CHARSET=latin1;

-- --------------------------------------------------------

--
-- Tabelstructuur voor tabel `homes`
--

CREATE TABLE IF NOT EXISTS `homes` (
  `id` int(11) NOT NULL,
  `name` varchar(64) NOT NULL,
  `country` varchar(2) NOT NULL,
  `city` varchar(64) NOT NULL,
  `zipcode` varchar(16) NOT NULL,
  `street` varchar(64) NOT NULL,
  `house_number` varchar(8) NOT NULL,
  `date_added` datetime(6) NOT NULL,
  `owner_id` int(11) NOT NULL
) ENGINE=InnoDB DEFAULT CHARSET=latin1;

-- --------------------------------------------------------

--
-- Tabelstructuur voor tabel `messages`
--

CREATE TABLE IF NOT EXISTS `messages` (
  `id` int(11) NOT NULL,
  `content` longtext NOT NULL,
  `plot` longtext NOT NULL,
  `date_sent` datetime(6) NOT NULL,
  `receiver_id` int(11) NOT NULL,
  `sender_id` int(11) NOT NULL
) ENGINE=InnoDB DEFAULT CHARSET=latin1;

-- --------------------------------------------------------

--
-- Tabelstructuur voor tabel `monthly_data`
--

CREATE TABLE IF NOT EXISTS `monthly_data` (
  `id` int(11) NOT NULL,
  `timestamp` datetime(6) NOT NULL,
  `usage` int(11) NOT NULL,
  `n_measurements` int(11) NOT NULL,
  `sensor_id` int(11) NOT NULL
) ENGINE=InnoDB DEFAULT CHARSET=latin1;

-- --------------------------------------------------------

--
-- Tabelstructuur voor tabel `recent_data`
--

CREATE TABLE IF NOT EXISTS `recent_data` (
  `id` int(11) NOT NULL,
  `timestamp` datetime(6) NOT NULL,
  `usage` int(11) NOT NULL,
  `n_measurements` int(11) NOT NULL,
  `sensor_id` int(11) NOT NULL
) ENGINE=InnoDB DEFAULT CHARSET=latin1;

-- --------------------------------------------------------

--
-- Tabelstructuur voor tabel `sensors`
--

CREATE TABLE IF NOT EXISTS `sensors` (
  `id` int(11) NOT NULL,
  `name` varchar(64) NOT NULL,
  `description` varchar(256) NOT NULL,
  `power_unit` varchar(3) NOT NULL,
  `date_created` datetime(6) NOT NULL,
  `home_id` int(11) NOT NULL
) ENGINE=InnoDB DEFAULT CHARSET=latin1;

-- --------------------------------------------------------

--
-- Tabelstructuur voor tabel `sensors_tags`
--

CREATE TABLE IF NOT EXISTS `sensors_tags` (
  `id` int(11) NOT NULL,
  `date_created` datetime(6) NOT NULL,
  `sensor_id` int(11) NOT NULL,
  `tag_id` int(11) NOT NULL
) ENGINE=InnoDB DEFAULT CHARSET=latin1;

-- --------------------------------------------------------

--
-- Tabelstructuur voor tabel `tags`
--

CREATE TABLE IF NOT EXISTS `tags` (
  `id` int(11) NOT NULL,
  `name` varchar(64) NOT NULL,
  `description` varchar(256) NOT NULL,
  `date_created` datetime(6) NOT NULL
) ENGINE=InnoDB DEFAULT CHARSET=latin1;

-- --------------------------------------------------------

--
-- Tabelstructuur voor tabel `users_homes`
--

CREATE TABLE IF NOT EXISTS `users_homes` (
  `id` int(11) NOT NULL,
  `permission_flags` int(11) NOT NULL,
  `date_created` datetime(6) NOT NULL,
  `home_id` int(11) NOT NULL,
  `user_id` int(11) NOT NULL
) ENGINE=InnoDB DEFAULT CHARSET=latin1;

-- --------------------------------------------------------

--
-- Tabelstructuur voor tabel `yearly_data`
--

CREATE TABLE IF NOT EXISTS `yearly_data` (
  `id` int(11) NOT NULL,
  `timestamp` datetime(6) NOT NULL,
  `usage` int(11) NOT NULL,
  `n_measurements` int(11) NOT NULL,
  `sensor_id` int(11) NOT NULL
) ENGINE=InnoDB DEFAULT CHARSET=latin1;

--
-- Indexen voor geëxporteerde tabellen
--

--
-- Indexen voor tabel `account_emailaddress`
--
ALTER TABLE `account_emailaddress`
  ADD PRIMARY KEY (`id`),
  ADD UNIQUE KEY `email` (`email`),
  ADD KEY `account_emailaddress_user_id_2c513194_fk_auth_user_id` (`user_id`);

--
-- Indexen voor tabel `account_emailconfirmation`
--
ALTER TABLE `account_emailconfirmation`
  ADD PRIMARY KEY (`id`),
  ADD UNIQUE KEY `key` (`key`),
  ADD KEY `account_ema_email_address_id_5b7f8c58_fk_account_emailaddress_id` (`email_address_id`);

--
-- Indexen voor tabel `authtoken_token`
--
ALTER TABLE `authtoken_token`
  ADD PRIMARY KEY (`key`),
  ADD UNIQUE KEY `user_id` (`user_id`);

--
-- Indexen voor tabel `auth_group`
--
ALTER TABLE `auth_group`
  ADD PRIMARY KEY (`id`),
  ADD UNIQUE KEY `name` (`name`);

--
-- Indexen voor tabel `auth_group_permissions`
--
ALTER TABLE `auth_group_permissions`
  ADD PRIMARY KEY (`id`),
  ADD UNIQUE KEY `auth_group_permissions_group_id_0cd325b0_uniq` (`group_id`,`permission_id`),
  ADD KEY `auth_group_permissi_permission_id_84c5c92e_fk_auth_permission_id` (`permission_id`);

--
-- Indexen voor tabel `auth_permission`
--
ALTER TABLE `auth_permission`
  ADD PRIMARY KEY (`id`),
  ADD UNIQUE KEY `auth_permission_content_type_id_01ab375a_uniq` (`content_type_id`,`codename`);

--
-- Indexen voor tabel `auth_user`
--
ALTER TABLE `auth_user`
  ADD PRIMARY KEY (`id`),
  ADD UNIQUE KEY `username` (`username`);

--
-- Indexen voor tabel `auth_user_groups`
--
ALTER TABLE `auth_user_groups`
  ADD PRIMARY KEY (`id`),
  ADD UNIQUE KEY `auth_user_groups_user_id_94350c0c_uniq` (`user_id`,`group_id`),
  ADD KEY `auth_user_groups_group_id_97559544_fk_auth_group_id` (`group_id`);

--
-- Indexen voor tabel `auth_user_user_permissions`
--
ALTER TABLE `auth_user_user_permissions`
  ADD PRIMARY KEY (`id`),
  ADD UNIQUE KEY `auth_user_user_permissions_user_id_14a6b632_uniq` (`user_id`,`permission_id`),
  ADD KEY `auth_user_user_perm_permission_id_1fbb5f2c_fk_auth_permission_id` (`permission_id`);

--
-- Indexen voor tabel `daily_data`
--
ALTER TABLE `daily_data`
  ADD PRIMARY KEY (`id`),
  ADD KEY `daily_data_d96d866a` (`sensor_id`);

--
-- Indexen voor tabel `friend_requests`
--
ALTER TABLE `friend_requests`
  ADD PRIMARY KEY (`id`),
  ADD KEY `friend_requests_receiver_id_6e370850_fk_auth_user_id` (`receiver_id`),
  ADD KEY `friend_requests_sender_id_e02b64d5_fk_auth_user_id` (`sender_id`);

--
-- Indexen voor tabel `homes`
--
ALTER TABLE `homes`
  ADD PRIMARY KEY (`id`),
  ADD KEY `homes_owner_id_32e87e52_fk_auth_user_id` (`owner_id`);

--
-- Indexen voor tabel `messages`
--
ALTER TABLE `messages`
  ADD PRIMARY KEY (`id`),
  ADD KEY `messages_receiver_id_874b4e0a_fk_auth_user_id` (`receiver_id`),
  ADD KEY `messages_sender_id_dc5a0bbd_fk_auth_user_id` (`sender_id`);

--
-- Indexen voor tabel `monthly_data`
--
ALTER TABLE `monthly_data`
  ADD PRIMARY KEY (`id`),
  ADD KEY `monthly_data_d96d866a` (`sensor_id`);

--
-- Indexen voor tabel `recent_data`
--
ALTER TABLE `recent_data`
  ADD PRIMARY KEY (`id`),
  ADD KEY `recent_data_d96d866a` (`sensor_id`);

--
-- Indexen voor tabel `sensors`
--
ALTER TABLE `sensors`
  ADD PRIMARY KEY (`id`),
  ADD KEY `sensors_home_id_1b6b9891_fk_homes_id` (`home_id`);

--
-- Indexen voor tabel `sensors_tags`
--
ALTER TABLE `sensors_tags`
  ADD PRIMARY KEY (`id`),
  ADD KEY `sensors_tags_sensor_id_ec6d5d5a_fk_sensors_id` (`sensor_id`),
  ADD KEY `sensors_tags_76f094bc` (`tag_id`);

--
-- Indexen voor tabel `tags`
--
ALTER TABLE `tags`
  ADD PRIMARY KEY (`id`);

--
-- Indexen voor tabel `users_homes`
--
ALTER TABLE `users_homes`
  ADD PRIMARY KEY (`id`),
  ADD KEY `users_homes_home_id_6d8097a2_fk_homes_id` (`home_id`),
  ADD KEY `users_homes_user_id_c35d6efe_fk_auth_user_id` (`user_id`);

--
-- Indexen voor tabel `yearly_data`
--
ALTER TABLE `yearly_data`
  ADD PRIMARY KEY (`id`),
  ADD KEY `yearly_data_sensor_id_349bf5e9_fk_sensors_id` (`sensor_id`);

--
-- AUTO_INCREMENT voor geëxporteerde tabellen
--

--
-- AUTO_INCREMENT voor een tabel `account_emailaddress`
--
ALTER TABLE `account_emailaddress`
  MODIFY `id` int(11) NOT NULL AUTO_INCREMENT;
--
-- AUTO_INCREMENT voor een tabel `account_emailconfirmation`
--
ALTER TABLE `account_emailconfirmation`
  MODIFY `id` int(11) NOT NULL AUTO_INCREMENT;
--
-- AUTO_INCREMENT voor een tabel `auth_group`
--
ALTER TABLE `auth_group`
  MODIFY `id` int(11) NOT NULL AUTO_INCREMENT;
--
-- AUTO_INCREMENT voor een tabel `auth_group_permissions`
--
ALTER TABLE `auth_group_permissions`
  MODIFY `id` int(11) NOT NULL AUTO_INCREMENT;
--
-- AUTO_INCREMENT voor een tabel `auth_permission`
--
ALTER TABLE `auth_permission`
  MODIFY `id` int(11) NOT NULL AUTO_INCREMENT,AUTO_INCREMENT=76;
--
-- AUTO_INCREMENT voor een tabel `auth_user`
--
ALTER TABLE `auth_user`
  MODIFY `id` int(11) NOT NULL AUTO_INCREMENT;
--
-- AUTO_INCREMENT voor een tabel `auth_user_groups`
--
ALTER TABLE `auth_user_groups`
  MODIFY `id` int(11) NOT NULL AUTO_INCREMENT;
--
-- AUTO_INCREMENT voor een tabel `auth_user_user_permissions`
--
ALTER TABLE `auth_user_user_permissions`
  MODIFY `id` int(11) NOT NULL AUTO_INCREMENT;
--
-- AUTO_INCREMENT voor een tabel `daily_data`
--
ALTER TABLE `daily_data`
  MODIFY `id` int(11) NOT NULL AUTO_INCREMENT;
--
-- AUTO_INCREMENT voor een tabel `friend_requests`
--
ALTER TABLE `friend_requests`
  MODIFY `id` int(11) NOT NULL AUTO_INCREMENT;
--
-- AUTO_INCREMENT voor een tabel `homes`
--
ALTER TABLE `homes`
  MODIFY `id` int(11) NOT NULL AUTO_INCREMENT;
--
-- AUTO_INCREMENT voor een tabel `messages`
--
ALTER TABLE `messages`
  MODIFY `id` int(11) NOT NULL AUTO_INCREMENT;
--
-- AUTO_INCREMENT voor een tabel `monthly_data`
--
ALTER TABLE `monthly_data`
  MODIFY `id` int(11) NOT NULL AUTO_INCREMENT;
--
-- AUTO_INCREMENT voor een tabel `recent_data`
--
ALTER TABLE `recent_data`
  MODIFY `id` int(11) NOT NULL AUTO_INCREMENT;
--
-- AUTO_INCREMENT voor een tabel `sensors`
--
ALTER TABLE `sensors`
  MODIFY `id` int(11) NOT NULL AUTO_INCREMENT;
--
-- AUTO_INCREMENT voor een tabel `sensors_tags`
--
ALTER TABLE `sensors_tags`
  MODIFY `id` int(11) NOT NULL AUTO_INCREMENT;
--
-- AUTO_INCREMENT voor een tabel `tags`
--
ALTER TABLE `tags`
  MODIFY `id` int(11) NOT NULL AUTO_INCREMENT;
--
-- AUTO_INCREMENT voor een tabel `users_homes`
--
ALTER TABLE `users_homes`
  MODIFY `id` int(11) NOT NULL AUTO_INCREMENT;
--
-- AUTO_INCREMENT voor een tabel `yearly_data`
--
ALTER TABLE `yearly_data`
  MODIFY `id` int(11) NOT NULL AUTO_INCREMENT;
--
-- Beperkingen voor geëxporteerde tabellen
--

--
-- Beperkingen voor tabel `account_emailaddress`
--
ALTER TABLE `account_emailaddress`
  ADD CONSTRAINT `account_emailaddress_user_id_2c513194_fk_auth_user_id` FOREIGN KEY (`user_id`) REFERENCES `auth_user` (`id`);

--
-- Beperkingen voor tabel `account_emailconfirmation`
--
ALTER TABLE `account_emailconfirmation`
  ADD CONSTRAINT `account_ema_email_address_id_5b7f8c58_fk_account_emailaddress_id` FOREIGN KEY (`email_address_id`) REFERENCES `account_emailaddress` (`id`);

--
-- Beperkingen voor tabel `authtoken_token`
--
ALTER TABLE `authtoken_token`
  ADD CONSTRAINT `authtoken_token_user_id_35299eff_fk_auth_user_id` FOREIGN KEY (`user_id`) REFERENCES `auth_user` (`id`);

--
-- Beperkingen voor tabel `auth_group_permissions`
--
ALTER TABLE `auth_group_permissions`
  ADD CONSTRAINT `auth_group_permissi_permission_id_84c5c92e_fk_auth_permission_id` FOREIGN KEY (`permission_id`) REFERENCES `auth_permission` (`id`),
  ADD CONSTRAINT `auth_group_permissions_group_id_b120cbf9_fk_auth_group_id` FOREIGN KEY (`group_id`) REFERENCES `auth_group` (`id`);

--
-- Beperkingen voor tabel `auth_permission`
--
ALTER TABLE `auth_permission`
  ADD CONSTRAINT `auth_permissi_content_type_id_2f476e4b_fk_django_content_type_id` FOREIGN KEY (`content_type_id`) REFERENCES `django_content_type` (`id`);

--
-- Beperkingen voor tabel `auth_user_groups`
--
ALTER TABLE `auth_user_groups`
  ADD CONSTRAINT `auth_user_groups_group_id_97559544_fk_auth_group_id` FOREIGN KEY (`group_id`) REFERENCES `auth_group` (`id`),
  ADD CONSTRAINT `auth_user_groups_user_id_6a12ed8b_fk_auth_user_id` FOREIGN KEY (`user_id`) REFERENCES `auth_user` (`id`);

--
-- Beperkingen voor tabel `auth_user_user_permissions`
--
ALTER TABLE `auth_user_user_permissions`
  ADD CONSTRAINT `auth_user_user_perm_permission_id_1fbb5f2c_fk_auth_permission_id` FOREIGN KEY (`permission_id`) REFERENCES `auth_permission` (`id`),
  ADD CONSTRAINT `auth_user_user_permissions_user_id_a95ead1b_fk_auth_user_id` FOREIGN KEY (`user_id`) REFERENCES `auth_user` (`id`);

--
-- Beperkingen voor tabel `daily_data`
--
ALTER TABLE `daily_data`
  ADD CONSTRAINT `daily_data_sensor_id_c70d4260_fk_sensors_id` FOREIGN KEY (`sensor_id`) REFERENCES `sensors` (`id`);

--
-- Beperkingen voor tabel `friend_requests`
--
ALTER TABLE `friend_requests`
  ADD CONSTRAINT `friend_requests_receiver_id_6e370850_fk_auth_user_id` FOREIGN KEY (`receiver_id`) REFERENCES `auth_user` (`id`),
  ADD CONSTRAINT `friend_requests_sender_id_e02b64d5_fk_auth_user_id` FOREIGN KEY (`sender_id`) REFERENCES `auth_user` (`id`);

--
-- Beperkingen voor tabel `homes`
--
ALTER TABLE `homes`
  ADD CONSTRAINT `homes_owner_id_32e87e52_fk_auth_user_id` FOREIGN KEY (`owner_id`) REFERENCES `auth_user` (`id`);

--
-- Beperkingen voor tabel `messages`
--
ALTER TABLE `messages`
  ADD CONSTRAINT `messages_receiver_id_874b4e0a_fk_auth_user_id` FOREIGN KEY (`receiver_id`) REFERENCES `auth_user` (`id`),
  ADD CONSTRAINT `messages_sender_id_dc5a0bbd_fk_auth_user_id` FOREIGN KEY (`sender_id`) REFERENCES `auth_user` (`id`);

--
-- Beperkingen voor tabel `monthly_data`
--
ALTER TABLE `monthly_data`
  ADD CONSTRAINT `monthly_data_sensor_id_5f3ceaa9_fk_sensors_id` FOREIGN KEY (`sensor_id`) REFERENCES `sensors` (`id`);

--
-- Beperkingen voor tabel `recent_data`
--
ALTER TABLE `recent_data`
  ADD CONSTRAINT `recent_data_sensor_id_4cb30b35_fk_sensors_id` FOREIGN KEY (`sensor_id`) REFERENCES `sensors` (`id`);

--
-- Beperkingen voor tabel `sensors`
--
ALTER TABLE `sensors`
  ADD CONSTRAINT `sensors_home_id_1b6b9891_fk_homes_id` FOREIGN KEY (`home_id`) REFERENCES `homes` (`id`);

--
-- Beperkingen voor tabel `sensors_tags`
--
ALTER TABLE `sensors_tags`
  ADD CONSTRAINT `sensors_tags_sensor_id_ec6d5d5a_fk_sensors_id` FOREIGN KEY (`sensor_id`) REFERENCES `sensors` (`id`),
  ADD CONSTRAINT `sensors_tags_tag_id_7ab85af9_fk_tags_id` FOREIGN KEY (`tag_id`) REFERENCES `tags` (`id`);

--
-- Beperkingen voor tabel `users_homes`
--
ALTER TABLE `users_homes`
  ADD CONSTRAINT `users_homes_home_id_6d8097a2_fk_homes_id` FOREIGN KEY (`home_id`) REFERENCES `homes` (`id`),
  ADD CONSTRAINT `users_homes_user_id_c35d6efe_fk_auth_user_id` FOREIGN KEY (`user_id`) REFERENCES `auth_user` (`id`);

--
-- Beperkingen voor tabel `yearly_data`
--
ALTER TABLE `yearly_data`
  ADD CONSTRAINT `yearly_data_sensor_id_349bf5e9_fk_sensors_id` FOREIGN KEY (`sensor_id`) REFERENCES `sensors` (`id`);
\end{lstlisting}
\end{document}
